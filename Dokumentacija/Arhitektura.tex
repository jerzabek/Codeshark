\chapter{Arhitektura i dizajn sustava}
		
		  \noindent Arhitekturu je moguće podijeliti na tri podsustava:
	\begin{itemize}
		\item 	Web poslužitelj
		\item 	Web aplikacija
		\item 	Baza podataka
	\end{itemize}

		\underline{\textit{Web preglednik}} je program koji korisniku omogućuje pregled web-stranica i multimedijskog sadržaja vezanog uz njih. Korisnik putem web preglednika šalje zahtjev web poslužitelju te mu web preglednik kao interpreter prevodi web-stranicu i njezin sadržaj u format koji je korisniku razumljiv. \\
		\underline{\textit{Web poslužitelj}} temelj je rada web aplikacije te je njegov glavni zadatak omogućiti komunikaciju klijenta s aplikacijom. Komunikacija se ostvaruje preko protokola HTTP (engl. \textit{Hyper Text Transfer Protocol}), koji služi za prijenos informacija na webu. Web aplikacija se pokreće preko poslužitelja koji joj prosljeđuje zahtjev od strane korisnika. \\
		\underline{\textit{Web aplikacija}} služi za obradu korisničkih zahtjeva. Ovisno o zahtjevu, web aplikacija tijekom obrade zahtjeva pristupa bazi podataka te korisniku vraća odgovor u obliku HTML (engl. \textit{HyperText Markup Language}) dokumenta koji se prikazuje preko web preglednika. \\
		Codeshark web aplikacija bazirana je na programskom jeziku Python-u za razvoj \textit{backend}-a, zajedno s JavaScript-om uz biblioteku React za razvoj \textit{frontend}-a. Kao razvojno okruženje koristio se Microsoft Visual Studio. \\
		Arhitektura sustava temelji se na konceptu MVC-a (Model-View-Controller). Karakteristika MVC koncepta je nezavisan razvoj pojedinih dijelova aplikacije što za posljedicu ima jednostavnije ispitivanje kao i jednostavno razvijanje i dodavanje novih svojstava u sustav. \\
		\noindent MVC koncept sastoji se od:
	
	\begin{itemize}
		\item \textbf{Model} - Središnja komponenta sustava. Predstavlja dinamičke strukture podataka, neovisne o korisničkom sučelju. Izravno upravlja podacima, logikom i pravilima aplikacije. Ujedno i prima ulazne podatke od Controller-a.
		\item \textbf{View} - Bilo kakav prikaz podataka, poput grafa. Mogući su različiti prikazi iste informacije poput grafičkog ili tabličnog prikaza podataka.
		\item \textbf{Controller} - Prima ulaze i prilagođava ih za prosljeđivanje Model-u ili View-u. Upravlja korisničkim zahtjevima i temeljem njih izvodi daljnju interakciju s ostalim elementima sustava.
	\end{itemize}
		

				
		\section{Baza podataka}
			
			\textbf{\textit{dio 1. revizije}}\\
			
		\textit{Potrebno je opisati koju vrstu i implementaciju baze podataka ste odabrali, glavne komponente od kojih se sastoji i slično.}
		
			\subsection{Opis tablica}
			

				\textit{Svaku tablicu je potrebno opisati po zadanom predlošku. Lijevo se nalazi točno ime varijable u bazi podataka, u sredini se nalazi tip podataka, a desno se nalazi opis varijable. Svjetlozelenom bojom označite primarni ključ. Svjetlo plavom označite strani ključ}
				
				
				\begin{longtblr}[
					label=none,
					entry=none
					]{
						width = \textwidth,
						colspec={|X[6,l]|X[6, l]|X[20, l]|}, 
						rowhead = 1,
					} %definicija širine tablice, širine stupaca, poravnanje i broja redaka naslova tablice
					\hline \multicolumn{3}{|c|}{\textbf{korisnik - ime tablice}}	 \\ \hline[3pt]
					\SetCell{LightGreen}IDKorisnik & INT	&  	Lorem ipsum dolor sit amet, consectetur adipiscing elit, sed do eiusmod  	\\ \hline
					korisnickoIme	& VARCHAR &   	\\ \hline 
					email & VARCHAR &   \\ \hline 
					ime & VARCHAR	&  		\\ \hline 
					\SetCell{LightBlue} primjer	& VARCHAR &   	\\ \hline 
				\end{longtblr}
				
				
			
			\subsection{Dijagram baze podataka}
				\textit{ U ovom potpoglavlju potrebno je umetnuti dijagram baze podataka. Primarni i strani ključevi moraju biti označeni, a tablice povezane. Bazu podataka je potrebno normalizirati. Podsjetite se kolegija "Baze podataka".}
			
			\eject
			
			
		\section{Dijagram razreda}
		
			\textit{Potrebno je priložiti dijagram razreda s pripadajućim opisom. Zbog preglednosti je moguće dijagram razlomiti na više njih, ali moraju biti grupirani prema sličnim razinama apstrakcije i srodnim funkcionalnostima.}\\
			
			\textbf{\textit{dio 1. revizije}}\\
			
			\textit{Prilikom prve predaje projekta, potrebno je priložiti potpuno razrađen dijagram razreda vezan uz \textbf{generičku funkcionalnost} sustava. Ostale funkcionalnosti trebaju biti idejno razrađene u dijagramu sa sljedećim komponentama: nazivi razreda, nazivi metoda i vrste pristupa metodama (npr. javni, zaštićeni), nazivi atributa razreda, veze i odnosi između razreda.}\\
			
			\textbf{\textit{dio 2. revizije}}\\			
			
			\textit{Prilikom druge predaje projekta dijagram razreda i opisi moraju odgovarati stvarnom stanju implementacije}
			
			
			
			\eject
		
		\section{Dijagram stanja}
			
			
			\textbf{\textit{dio 2. revizije}}\\
			
			\textit{Potrebno je priložiti dijagram stanja i opisati ga. Dovoljan je jedan dijagram stanja koji prikazuje \textbf{značajan dio funkcionalnosti} sustava. Na primjer, stanja korisničkog sučelja i tijek korištenja neke ključne funkcionalnosti jesu značajan dio sustava, a registracija i prijava nisu. }
			
			
			\eject 
		
		\section{Dijagram aktivnosti}
			
			\textbf{\textit{dio 2. revizije}}\\
			
			 \textit{Potrebno je priložiti dijagram aktivnosti s pripadajućim opisom. Dijagram aktivnosti treba prikazivati značajan dio sustava.}
			
			\eject
		\section{Dijagram komponenti}
		
			\textbf{\textit{dio 2. revizije}}\\
		
			 \textit{Potrebno je priložiti dijagram komponenti s pripadajućim opisom. Dijagram komponenti treba prikazivati strukturu cijele aplikacije.}