\chapter{Specifikacija programske potpore}
		
	\section{Funkcionalni zahtjevi}
			
			\noindent \textbf{Dionici:}
			
			\begin{packed_enum}
				
				\item Natjecatelj
				\item Voditelj
				\item Administrator				
				\item Razvojni tim
				
			\end{packed_enum}
			
			\noindent \textbf{Aktori i njihovi funkcionalni zahtjevi:}
			
			
			\begin{packed_enum}
				\item  \underbar{Neregistrirani/neprijavljeni korisnik (inicijator) može:}
				
				\begin{packed_enum}
					
					\item pretraživati natjecanja i pristupiti popisu zadataka s natjecanja i sudionika natjecanja
					\item pretraživati i čitati zadatke s natjecanja
					\item pristupiti informacijama tuđeg korisničkog profila:
					\begin{packed_enum}
						
						\item  korisničko ime, ime i prezime, i osvojene nagrade
						\item  statistike u što se ubraja broj točno riješenih i broj isprobanih zadataka
						\item  popis natjecanja na kojima je korisnik prisustvovao ili natjecanja koja je održavao
				
					\end{packed_enum}
					\item se registrirati u sustav, stvoriti novi korisnički račun za koji:
					\begin{packed_enum}
						
						\item  su mu potrebni korisničko ime, lozinka, ime i prezime te e-mail adresa
						\item  mora izabrati razinu pristupa "natjecatelj" ili "voditelj"
						
					\end{packed_enum}
				\end{packed_enum}
			
				\item  \underbar{Natjecatelj (inicijator) može:}
				
				\begin{packed_enum}
					
					\item pregledavati i mijenjati osobne podatke
					\item izbrisati svoj korisnički račun
					\item pristupiti zadatcima za vježbu i rješavati ih
					\item prijaviti i odjaviti se na natjecanja i pristupiti natjecanjima
					\item osvojiti nagrade
					\item pokrenuti virtualno natjecanje
					
				\end{packed_enum}
			
					\item  \underbar{Voditelj (inicijator) može:}
				
				\begin{packed_enum}
					
					\item pokrenuti i voditi vlastito natjecanje 
					
				\end{packed_enum}
			
				\item  \underbar{Administrator (inicijator) može:}
				
				\begin{packed_enum}
					
					\item vidjeti popis svih registriranih korisnika i njihovih osobnih podataka
					\item korisnike brisati i mijenjati im razinu pristupa aplikaciji (natjecatelj, voditelj)
					\item pristupiti statistici
					\item stvarati i brisati zadatke
					\item odobravati zahtjeve za razinu pristupa voditelja
					
				\end{packed_enum}
			\end{packed_enum}
			
			\eject 
			
			
				
			\subsection{Obrasci uporabe}
				
				\textbf{\textit{dio 1. revizije}}
				
				\subsubsection{Opis obrazaca uporabe}
					\textit{Funkcionalne zahtjeve razraditi u obliku obrazaca uporabe. Svaki obrazac je potrebno razraditi prema donjem predlošku. Ukoliko u nekom koraku može doći do odstupanja, potrebno je to odstupanje opisati i po mogućnosti ponuditi rješenje kojim bi se tijek obrasca vratio na osnovni tijek.}\\
					

					\noindent \underbar{\textbf{UC$<$broj obrasca$>$ -$<$ime obrasca$>$}}
					\begin{packed_item}
	
						\item \textbf{Glavni sudionik: }$<$sudionik$>$
						\item  \textbf{Cilj:} $<$cilj$>$
						\item  \textbf{Sudionici:} $<$sudionici$>$
						\item  \textbf{Preduvjet:} $<$preduvjet$>$
						\item  \textbf{Opis osnovnog tijeka:}
						
						\item[] \begin{packed_enum}
	
							\item $<$opis korak jedan$>$
							\item $<$opis korak dva$>$
							\item $<$opis korak tri$>$
							\item $<$opis korak četiri$>$
							\item $<$opis korak pet$>$
						\end{packed_enum}
						
						\item  \textbf{Opis mogućih odstupanja:}
						
						\item[] \begin{packed_item}
	
							\item[2.a] $<$opis mogućeg scenarija odstupanja u koraku 2$>$
							\item[] \begin{packed_enum}
								
								\item $<$opis rješenja mogućeg scenarija korak 1$>$
								\item $<$opis rješenja mogućeg scenarija korak 2$>$
								
							\end{packed_enum}
							\item[2.b] $<$opis mogućeg scenarija odstupanja u koraku 2$>$
							\item[3.a] $<$opis mogućeg scenarija odstupanja  u koraku 3$>$
							
						\end{packed_item}
					\end{packed_item}
				
					
				\subsubsection{Dijagrami obrazaca uporabe}
					
					\textit{Prikazati odnos aktora i obrazaca uporabe odgovarajućim UML dijagramom. Nije nužno nacrtati sve na jednom dijagramu. Modelirati po razinama apstrakcije i skupovima srodnih funkcionalnosti.}
				\eject		
				
			\subsection{Sekvencijski dijagrami}
				
				\textbf{\textit{dio 1. revizije}}\\
				
				\textit{Nacrtati sekvencijske dijagrame koji modeliraju najvažnije dijelove sustava (max. 4 dijagrama). Ukoliko postoji nedoumica oko odabira, razjasniti s asistentom. Uz svaki dijagram napisati detaljni opis dijagrama.}
				\eject
	
		\section{Ostali zahtjevi}
		
			\textbf{\textit{dio 1. revizije}}\\
		 
			 \textit{Nefunkcionalni zahtjevi i zahtjevi domene primjene dopunjuju funkcionalne zahtjeve. Oni opisuju \textbf{kako se sustav treba ponašati} i koja \textbf{ograničenja} treba poštivati (performanse, korisničko iskustvo, pouzdanost, standardi kvalitete, sigurnost...). Primjeri takvih zahtjeva u Vašem projektu mogu biti: podržani jezici korisničkog sučelja, vrijeme odziva, najveći mogući podržani broj korisnika, podržane web/mobilne platforme, razina zaštite (protokoli komunikacije, kriptiranje...)... Svaki takav zahtjev potrebno je navesti u jednoj ili dvije rečenice.}
			 
			 
			 
	